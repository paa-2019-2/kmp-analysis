\section{Análise de complexidade do Merge Sort}
\label{complexity}
A complexidade de um algoritmo de ordenação se baseia em contar 
o número de comparações realizadas durante sua execução. 
Neste sentido, existe uma classificação de algoritmos de ordenação chamada
algoritmos de ordenação por comparação, que, segundo Cormen \cite{cormen2012algoritmos},
são algoritmos baseados em comparações entre dois elementos e 
efetuam pelo menos \textbf{\textit{$\Omega$(n lg n)}} comparações
no pior caso. 

O Merge Sort é um algoritmo de ordenação por comparação, e portanto,
sua análise de complexidade se baseia em contar quantas comparações
são efetuadas para ordenar uma lista de tamanho \textit{\textbf{n}}. Para
tanto, foram necessárias modificações no algoritmo, de forma a incluir
um contador de comparações. O pseudo-código abaixo representa a ideia 
da inclusão de um contador no Merge Sort:
$\newline$
\begin{algorithm}[H]
  \KwData{$\rho, \eta : Sorted\ lists,\ \sigma : natural $}
  \KwResult{$Sorted\ merge\ of\ \rho\ and\ \eta,\ number\ of\ comparisons$}
  \eIf{$\rho = Nil$ or $\eta = Nil$}{
      $\textbf{return}\  \rho + \eta,\ \sigma$ 
  }{
      \eIf{$car(\rho) \leq car(\eta)$}{
          $\textbf{return}\ car(\rho) + MERGE(cdr(\rho) ,\ \eta,\ \sigma + 1)$
      }{
          $\textbf{return}\ car(\eta) + MERGE(\rho,\ cdr(\eta),\ \sigma + 1)$
      }
  }
  \caption{COUNT-MERGE}
\end{algorithm}
$\newline$
$\newline$
\begin{algorithm}[H]
  \KwData{$\tau : List\ of\ comparable\ elements$}
  \KwResult{$Sorted\ permutation\ of\ \tau,\ number\ of\ comparisons $}
  \eIf{$length(\tau) \leq 1$}{
          $\textbf{return}\ \tau, \ 0 $
  }{
          $prefix,\ \sigma \leftarrow\ MERGESORT(first\_half(\tau))$\;
          $suffix,\ \iota \leftarrow\ MERGESORT(second\_half(\tau))$\;
          $\textbf{return}\ MERGE(prefix,\ suffix,\ \sigma + \iota)$\;  
  }
  \caption{COUNT-MERGESORT}
\end{algorithm}
$\newline$
$\newline$
A adição de um contador no Merge Sort permite a análise de comparações, porém,
não é possível assegurar que o contador não inseriu um erro no algoritmo. Portanto,
é necessário provar os seguintes lemas, que asseguram a equivalência entre o 
Merge Sort com e sem contador:

\begin{lemma}
    \label{cmerge_equiv_merge}
    $\forall$ $\rho$, $\eta$ listas de naturais, a resultante de $MERGE(\rho,\ \eta)$
    é igual à resultante de $COUNT-MERGE(\rho,\ \eta, \sigma)$, a não ser por $\sigma$.
\end{lemma}

\begin{proof}
  A ideia da prova é mostrar que a lista retornada por $MERGE$ é 
  igual à lista retornada por $COUNT-MERGE$. Utilizando indução 
  forte sobre a soma tamanho das listas $\rho$ e $\eta$, 
  temos que para quaisquer listas cujo a soma dos tamanhos das mesmas
  é menor que a soma do tamanho das listas, a propriedade vale. Portanto
  basta expandir a definição de $MERGE$ e $COUNT-MERGE$ e analisar
  os seguintes casos, no momento da chamada recursiva:
    \begin{enumerate}
      \item cabeça de $\rho$ é maior ou igual que a cabeça de $\eta$
      \item a cabeça de $\rho$ é menor que a cabeça de $\eta$
    \end{enumerate}
  Em ambos, temos uma construção de uma lista no qual podemos ver que a 
  propriedade da indução pode ser aplicada, pois temos que tamanho da 
  cauda de $\rho$ + tamanho de $\eta$ é menor que tamanho de $\rho$ + tamanho
  de $\eta$, e precisamos mostrar que inserir a cabeça de $\rho$ na lista 
  construída resulta na mesma lista para as duas funções. De fato, ao aplicar
  a hipótese de indução, temos que inserir o mesmo elemento em duas listas iguais
  resulta em duas listas iguais, o que é trivialmente verdade.
\end{proof}

\begin{lemma}
  \label{cmergesort_equiv_mergesort}
  $\forall$ $\rho$, lista de naturais, a resultante de $MERGESORT(\rho)$
  é igual à resultante de $COUNT-MERGESORT(\rho)$, a não ser pelo contador retornado.
\end{lemma}

\begin{proof}
  De forma semelhante ao lema anterior, a ideia da prova é mostrar que o retorno
  de $MERGESORT$ e $COUNT-MERGESORT$ são equivalentes. Com indução forte, temos
  temos que para quaisquer listas cujo a soma dos tamanhos das mesmas é
  menor que a soma do tamanho das listas, a propriedade vale. Com isso, aplicamos
  a hipótese de indução para \textbf{\textit{prefix}} e \textbf{\textit{suffix}} e 
  expandimos as definições de $MERGESORT$ e $COUNT-MERGESORT$. Basta aplicar
  o lema anterior para $MERGE$ e $COUNT-MERGE$ na expansão de $MERGESORT$ e 
  $COUNT-MERGESORT$, e o sequente é provado.
\end{proof}

O lema a seguir é um lema auxiliar, necessário para a prova do lema \ref{merge_sort_ws}.

\begin{lemma}
  \label{cmerge_sort_length}
  $\forall$ $\rho$, lista de naturais, o tamanho da 
  resultante de $COUNT-MERGESORT$ é igual ao tamanho de $\rho$.
\end{lemma}

\begin{proof}
  Para tal, basta utilizar o lema \ref{cmergesort_equiv_mergesort} e o lema
  \ref{mergesort-preserves-length}.
\end{proof}

Os lemas seguintes são acerca da complexidade do Merge Sort, utilizando
os contadores de comparações. Os lemas \ref{count_cmerge_ws} e \ref{cmerge_general_case}
são acerca da complexidade da rotina $MERGE$. Já o lema \ref{merge_sort_ws} 
busca mostrar a complexidade de $MERGESORT$ por completo.

\begin{lemma}
  \label{count_cmerge_ws}
  $\forall \rho, \eta$, listas de naturais, o valor do contador 
  de comparações retornado pela chamada $COUNT-MERGE(\rho,\ \eta,\ 0)$ é 
  menor ou igual que \textbf{\textit{n + m}}, onde \textbf{\textit{n}}
  é o tamanho de $\rho$ e \textbf{\textit{m}} é o tamanho de $\eta$
\end{lemma}

\begin{proof}
  Primeiramente, utilizamos indução forte no tamanho das listas
  $\rho$ e $\eta$, e temos, como hipótese de indução,
  que a propriedade do lema vale para quaisquer listas tal 
  que a soma dos tamanhos dessas listas é menor que a soma 
  dos tamanhos de $\rho$ e $\eta$. Então, basta expandir $COUNT-MERGE$
  e analisar cada um dos casos:
  \begin{enumerate}
    \item Temos que provar que a soma dos tamanhos de 
      $\rho$ e $\eta$ é maior ou igual que 0, dado que $\rho$ é nulo
      ou $\eta$ é nulo.

    \item Temos que provar que 1 + $\sigma$ é menor ou igual a 
      tamanho de $\rho$ + tamanho de $\eta$, onde $\sigma$ é o contador
      retornado pela chamada $COUNT-MERGE(cdr(\rho),\ \eta,\ 0)$.

    \item Temos que provar que 1 + $\sigma$ é menor ou igual a 
      tamanho de $\rho$ + tamanho de $\eta$, onde $\sigma$ é o contador
      retornado pela chamada $COUNT-MERGE(\rho,\ cdr(\eta),\ 0)$.
  \end{enumerate}

  Para o item 1, basta verificar que o tamanho não pode ser negativo,
  então é sempre maior ou igual a zero.

  Para o item 2, utilizamos a hipótese de indução para tamanho de $cdr(\rho)$ +
  tamanho de $\eta$ e expandimos a definição de tamanho de listas não-nulas
  para verificar que a propriedade é verificada.

  Para o item 3, basta utilizar a mesma ideia da prova do item 2, com
  a hipótese de indução sendo utilizada para $cdr(\eta)$.
\end{proof}

\begin{lemma}
  \label{cmerge_general_case}
  $\forall \rho, \eta$, listas de naturais, e $\sigma$, um natural,
  o valor do contador de comparações retornado pela chamada 
  $COUNT-MERGE(\rho,\ \eta,\ \sigma)$ é menor ou igual que 
  \textbf{\textit{n + m + c}}, onde \textbf{\textit{n}} é o 
  tamanho de $\rho$, \textbf{\textit{m}} é o tamanho de $\eta$ e
  \textbf{\textit{c}} é o número de comparações realizadas anteriormente, 
  ou seja, $\sigma$.
\end{lemma}

\begin{proof}
  Por ser uma generalização do lema anterior, a prova desse lema se
  dá de forma bem semelhante. Começamos utilizando indução forte em 
  $LENGTH(\rho)$ + $LENGTH(\eta)$ e temos como hipótese que a propriedade
  vale para quaisquer listas tal que a soma dos tamanhos 
  dessas listas é menor que a soma dos tamanhos de $\rho$ e $\eta$.
  Ao expandir a definição de $COUNT-MERGE$ temos os casos:

  \begin{enumerate}
    \item $\sigma$ é menor ou igual a $LENGTH(\rho)$ + $LENGTH(\eta)$ 
      + $\sigma$, onde $\sigma$ é o contador retornado pela
      chamada $COUNT-MERGE(\rho,\ \eta,\ \sigma)$.
     
  \item 1 + $\sigma$ é menor ou igual a 
    $LENGTH(\rho)$ + $LENGTH(\eta)$ + $\sigma$, onde $\sigma$ é o contador
    retornado pela chamada $COUNT-MERGE(cdr(\rho),\ \eta,\ \sigma)$.

  \item 1 + $\sigma$ é menor ou igual a $LENGTH(\rho)$ + 
    $LENGTH(\eta)$ + $\sigma$, onde $\sigma$ é o contador
    retornado pela chamada $COUNT-MERGE(\rho,\ cdr(\eta),\ \sigma)$.
\end{enumerate}

Para o item 1, assim como no lema anterior, temos que o tamanho de uma
lista não pode ser negativo, portanto, o sequente é trivialmente verdade.

Para o item 2, utilizamos a hipótese de indução para $cdr(\rho)$ e 
expandirmos a definição de LENGTH, que mostra que o sequente é verdade.

Para o item 3, basta utilizar a ideia da prova do item 2, porém utilizando
a hipótese de indução para $cdr(\eta)$.
\end{proof}

O lema asseguir representa a complexidade de fato do algoritmo,
e utiliza o seguinte axioma, também utilizado por Cormen\cite{cormen2012algoritmos}:

\begin{axiom}
  \label{floor_aux}
  $\forall \rho$, lista de naturais, floor(LENGTH($\rho$)/2) 
  = LENGTH($\rho$)/2, onde
  \textbf{floor} denota o arrendondamento para baixo. 
\end{axiom}

\begin{lemma}
  \label{merge_sort_ws}
  $\forall \rho$, lista de naturais, tal que tamanho de $\rho$ é maior
  que zero, o valor do contador de comparações retornado pela chamada 
  $COUNT-MERGESORT(\rho)$ é menor ou igual a \textbf{n + n log n}, onde
  \textbf{n} é o tamanho de $\rho$.
\end{lemma}

\begin{proof}
  A ideia central da prova é utilizar substituições para
  chegar no resultado final. Para tanto, começamos a prova utilizando
  indução forte sobre LENGTH($\rho$), sendo a hipótese de indução
  que a propriedade vale para listas cujo tamanho é menor que o tamanho
  de $\rho$, e expandimos a definição de $COUNT-MERGESORT$. 
  Usamos o lema \ref{cmerge_general_case} e o lema \ref{cmerge_sort_length}
  para chegar ao sequente: o contador retornado por $COUNT-MERGE$ 
  aplicada a $COUNT-MERGESORT$ expandido é LENGTH($\rho$) + 
  $\sigma + \iota$, onde $\sigma$ é o valor do contador de $COUNT-MERGESORT$
  aplicado ao prefixo de $\rho$ e $\iota$ é o valor do contador de $COUNT-MERGESORT$
  aplicado ao sufixo de $\rho$, ou seja, chegamos à equação de recorrência
  T(n) $\le$ T($\ceil{\frac{n}{2}}$) + T($\floor{\frac{n}{2}}$) + n, onde \textbf{n}
  é LENGTH($\rho$).

  Com a equação de recorrência presente no sequente, utilizamos a hipótese 
  de indução para suffix e prefix, e o axioma \ref{floor_aux}, 
  que simplifica a recorrência e a transforma em T(n) $\le$ 
  2 * T($\frac{n}{2}$) + n. Basta utilizar então as propriedades de
  divisão e inverso do logaritmo para provar o sequente.
\end{proof}

Sendo assim, a complexidade do Merge Sort é limitada por 
\textit{\textbf{n + n lg n}}. Falta então, definir qual o limite
superior em Omicron.