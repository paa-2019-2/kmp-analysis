\section{Análise de complexidade do Merge Sort}
\label{complexity}
A complexidade de um algoritmo de ordenação se baseia em contar 
o número de comparações realizadas durante sua execução. 
Neste sentido, existe uma classificação de algoritmos de ordenação chamada
algoritmos de ordenação por comparação, que, segundo Cormen \cite{cormen2012algoritmos},
são algoritmos baseados em comparações entre dois elementos e 
efetuam pelo menos \textbf{\textit{$\Omega$(n lg n)}} comparações
no pior caso. 

O Merge Sort é um algoritmo de ordenação por comparação, e portanto,
sua análise de complexidade se baseia em contar quantas comparações
são efetuadas para ordenar uma lista de tamanho \textit{\textbf{n}}. Para
tanto, foram necessárias modificações no algoritmo, de forma a incluir
um contador de comparações. O pseudo-código abaixo representa a ideia 
da inclusão de um contador no Merge Sort:
$\newline$
\begin{algorithm}[H]
  \KwData{$\rho, \eta : Sorted\ lists,\ \sigma : natural $}
  \KwResult{$Sorted\ merge\ of\ \rho\ and\ \eta,\ number\ of\ comparisons$}
  \eIf{$\rho = Nil$ or $\eta = Nil$}{
      $\textbf{return}\  \rho + \eta,\ \sigma$ 
  }{
      \eIf{$car(\rho) \leq car(\eta)$}{
          $\textbf{return}\ car(\rho) + MERGE(cdr(\rho) ,\ \eta,\ \sigma + 1)$
      }{
          $\textbf{return}\ car(\eta) + MERGE(\rho,\ cdr(\eta),\ \sigma + 1)$
      }
  }
  \caption{COUNT-MERGE}
\end{algorithm}
$\newline$
$\newline$
\begin{algorithm}[H]
  \KwData{$\tau : List\ of\ comparable\ elements$}
  \KwResult{$Sorted\ permutation\ of\ \tau,\ number\ of\ comparisons $}
  \eIf{$length(\tau) \leq 1$}{
          $\textbf{return}\ \tau, \ 0 $
  }{
          $prefix,\ \sigma \leftarrow\ MERGESORT(first\_half(\tau))$\;
          $suffix,\ \iota \leftarrow\ MERGESORT(second\_half(\tau))$\;
          $\textbf{return}\ MERGE(prefix,\ suffix,\ \sigma + \iota)$\;  
  }
  \caption{COUNT-MERGESORT}
\end{algorithm}
$\newline$
$\newline$
A adição de um contador no Merge Sort permite a análise de comparações, porém,
não é possível assegurar que o contador não inseriu um erro no algoritmo. Portanto,
é necessário provar os seguintes lemas, que asseguram a equivalência entre o 
Merge Sort com e sem contador:

\begin{lemma}
    \label{cmerge_equiv_merge}
    $\forall$ $\rho$, $\eta$ listas de naturais, a resultante de $MERGE(\rho,\ \eta)$
    é igual à resultante de $COUNT-MERGE(\rho,\ \eta, \sigma)$, a não ser por $\sigma$.
\end{lemma}

\begin{lemma}
  \label{cmergesort_equiv_mergesort}
  $\forall$ $\rho$, lista de naturais, a resultante de $MERGESORT(\rho)$
  é igual à resultante de $COUNT-MERGESORT(\rho)$, a não ser pelo contador retornado.
\end{lemma}