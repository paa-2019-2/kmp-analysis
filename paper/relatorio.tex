\documentclass[12pt]{article}

\usepackage[utf8]{inputenc}
\usepackage[brazil]{babel}
\usepackage{amsmath,amssymb}
\usepackage{pdfsync}
\usepackage[all]{xy}
\usepackage{color}
\usepackage{hyperref}
\usepackage{graphicx}
\usepackage[ruled,vlined]{algorithm2e}
\usepackage{amsthm}

\newtheorem{theorem}{Teorema}[section]
\newtheorem{corollary}{Corolário}[theorem]
\newtheorem{lemma}[theorem]{Lema}
\theoremstyle{definition}
\newtheorem{definition}{Definição}[section]

\title{{\large Universidade de Brasília \\ Instituto de Ciências Exatas \\
Departamento de Ciência da Computação} \\[1cm]
117536 - Projeto e Análise de Algoritmos Turma: B\\[.5cm]
Análise Assintótica e Corretude do Algoritmo MergeSort Utilizando PVS}
\author{Gabriel Levi - 16/0006490 \\
        Gabriel Nunes - 16/0006597}
\date{\today}

\begin{document}
\maketitle
\newpage

\section{Introdução}
\noindent A verificação formal de algoritmo, no que tange ao seu comportamento assintótico e a corretude do algoritmo é interesse
central da Ciência da Computação. Por vezes, a prova via argumentação, isto é, lápis e papel pode ser suficiente para que o escritor
convença o leitor de que o algoritmo está correto. Contudo, esse modelo de prova se sustenta muita vezes em passos de pura intuição,
assumições que ambas as partes enxergam como um axioma, e saltos lógicos que por mais naturais que pareçam escondem um conjunto não-trivial
de conceitos. A ocorrência desses aspectos em uma formalização pode ocultar falhas que, de fato, provem a incorretude do algoritmo e desmonstre
um comportamento assintótico pior do que o esperado.

Como forma de minimizar o exposto anteriormente, introduz-se os sistemas de verificação de provas. Os verificadores garantem que os passos realizados
dentro de uma prova respeitem o conjunto de regras de sua lógica intrínseca. Então, dadas premissas corretas e um ponto factível onde se deseja chegar, qualquer
passo intermediário tem que, necessariamente, estar correto. Obviamente, construções ruins de objetivos e premissas podem levar a provas, ainda sim, incorretas
ou impossíveis. Podemos concluir então que os sistemas de verificação pressupõe que uma prova, ou pelo menos a ideia da mesma, já exista e o usuário interessado
o utilize para demonstrar que de fato aquela construção vale.

Um dos verificadores, PVS - Prototype Verification System - é a linguagem de especificação e provador automatizado de teoremas que aqui será utilizado.
PVS trabalha com implementações em distribuições de diferentes versões de LISP. Em um arquivo, o usuário define premissas - um algoritmo - e teoremas.
O arquivo é dado como entrada para o provador que requisita as regras a serem aplicadas até que o teorema desejado seja provado. As regras reconhecidas pelo
PVS tratam-se regras do cálculo de sequentes. O PVS permite também que uma prova possa ser revisitada em uma representação gráfica que
revela cada passo bem como suas depedências.

O objetivo deste trabalho é analisar assintóticamente o custo de tempo e a corretude do algoritmo de ordenação \textit{Merge Sort} via PVS. O Merge Sort foi criado
em meados de 1945 por John Von Neumann. A escolha deste algoritmo se deu por ser de implementação muito simples para múltiplas estruturas de dados tais como
vetores e listas ligadas e por, ainda sim, ser um algoritmo com múltiplas aplicações dado o seu custo de execução. Este relatório apresentará a ideia de prova
formalizada via provador e fica a critério do leitor visitar o repositório para verificar a mesma. A formalização completa pode ser encontrada em \url{github.com/paa-2019-2/levi-nunes}.

Esse documento se organizará, daqui em diante, por um capítulo \ref{explaining} de revisão teórica. Em seguida, no capitulo \ref{merge_sort}, a apresentação do algoritmo Merge Sort. 
O capitulo \ref{complexity} apresentará a ideia abordada pelos autores para a análise de assintótica do algoritmo
bem como argumentação sobre o pior e o melhor caso do mesmo. O capitulo \ref{correctness} apresentará, como o capitulo anterior, a ideia abordada
e sua argumentação. Por fim, o capitulo de conclusão, apresentará um resumo dos resultados aqui encontrados acompanhado de um meta-texto discorrendo sobre
as principais dificuldades do trabalho com o verificador formal de provas.

\section{Revisão teórica}
\label{explaining}

Este capitulo introduzirá conceitos necessários e notações necessárias para o entedimento das seções posteriores.

\subsection{Notação assintótica}

A notação assintótica é uma forma de descrever o comportamento de uma função matemática para um $n$ inteiros suficientemente
grandes. Neste trabalho, interessa-se somente funções eventualmente crescentes uma vez que a intenção é analisar custo de execucação
de algoritmos. Segue-se então as definições:

\theoremstyle{definition}
\begin{definition}
        $O(g(n)) := \{f(n)\ |\ 0 \leq f(n) \leq c * g(n), c\ constante\ e\ n \geq n_0\}$        
\end{definition}

Uma função $f(n)$ que descreve o custo de um algoritmo A pertence à $O(g(n))$ se, e somente se, para entradas 
suficientemente grandes o custo do pior caso do algoritmo é limitado superiormente por uma função da ordem de $g(n)$.
Isto quer dizer independente da entrada $n$, $n \geq n_0$  o algoritmo nunca tomará mais que $c * g(n)$ passos, $c$ constante, para concluir
sua execução.

\theoremstyle{definition}
\begin{definition}
        $\Omega(g(n)) := \{f(n)\ |\ 0 \leq c * g(n) \leq f(n), c\ constante\ e\ n \geq n_0\}$        
\end{definition}

A definição de $\Omega$ é similar a definição de $O$, com a diferença que $\Omega$ descreve o comportamento do algoritmo
em seu melhor caso. Isto é, independente da entrada $n$, $n \geq n_0$ o algoritmo tomará ao menos $c * g(n)$ passos, $c$ constante, para concluir
sua execução.

\theoremstyle{definition}
\begin{definition}
        $\Theta(g(n)) := \{f(n)\ | f(n) \in \Omega(g(n)) \wedge f(n) \in O(g(n))\}$        
\end{definition}

Por último, a definição de $\Theta$ descreve uma classe de algoritmos em que o melhor e o pior caso estão na mesma ordem
de complexidade. É uma notação mais precisa em relação a complexidade do algoritmo mas que, efetivamente, não se aplica
a qualquer análise.

\subsection{Lista}

Uma lista é uma estrutura recursiva que obedece a seguinte gramática:

\[list := nil\ |\ a::list\]

Onde $nil$ representa uma lista vazia, $a::list$ representa um elemento qualquer concatenado com uma lista.
À fim de simplicidade, o primeiro elemento de uma lista $l$ é denotado aqui como $car(l)$ e a lista restante,
a cauda, é denotada como $cdr(l)$. Também para fim de simplicidade, denota-se para uma lista $l$, qualquer, $LENGTH(l)$ 
a quantidade de elementos que a mesma armazena. 

\section{O algoritmo Merge Sort}
\label{merge_sort}

O algoritmo fundamenta-se na técnica Dividir-para-Conquistar. A técnica consiste em, dada uma instância do problema
quebra-la em partes até que as mesmas sejam fáceis ou triviais de se resolver, por fim, cada pequena solução é combinada para
de obter a solução para o problema maior. Apesar de parecer similar, esta técnica é bem diferente de programação dinâmica pois
cada subproblema é disjunto dos demais. Desta forma, o algoritmo recebe uma lista como entrada uma lista de elementos comparáveis
e divide tal lista até que cada sublista possua tamanho unitário - trivialmente ordenável - e então combina as listas para obter
a lista de entrada ordenada. Por mais poderosa que seja a ideia, a implementação do Merge Sort é bem simples e pode ser deduzida
a partir dos pseudo-códigos \ref{merge_pseudocode} responsável pela combinação das subsoluções e \ref{mergesort_pseudocode} responsável pelo processo de
divisão.

\begin{algorithm}[H]
\label{merge_pseudocode}
        \KwData{$\rho, \eta : Sorted\ lists $}
        \KwResult{Sorted merge of $\rho$ and $\eta$}
        \eIf{$\rho = Nil$ or $\eta = Nil$}{
            $\textbf{return}\  \rho + \eta$
        }{
            \eIf{$car(\rho) \leq car(\eta)$}{
                $\textbf{return}\ car(\rho) + MERGE(cdr(\rho) ,\ \eta)$
            }{
                $\textbf{return}\ car(\eta) + MERGE(\rho,\ cdr(\eta))$
            }
        }
        \caption{MERGE}
\end{algorithm}

\begin{algorithm}[H]
\label{mergesort_pseudocode}
        \KwData{$\tau : List\ of\ comparable\ elements $}
        \KwResult{$Sorted\ permutation\ of\ \tau$}
        \eIf{$length(\tau) \leq 1$}{
                $\textbf{return}\ \tau$
        }{
                $prefix \leftarrow\ MERGESORT(first\_half(\tau))$\;
                $suffix \leftarrow\ MERGESORT(second\_half(\tau))$\;
                $\textbf{return}\ MERGE(prefix,\ suffix)$\;  
        }
        \caption{MERGESORT}
\end{algorithm}

\section{Corretude do algoritmo Merge Sort}
\label{correctness}

A corretude de um algoritmo passa por demonstrar que o mesmo possui certas características independente da instância e que essas características
se verifiquem antes, durante e depois da execução. Para um algoritmo de ordenação, é esperado que o mesmo responda para qualquer entrada uma permutação
ordenada da mesma, isto é, a saída não só deve estar ordenada como também o número de ocorrências de cada elemento deve ser o mesmo da lista de entrada.

A análise de MERGESORT nos leva a uma série de resultados intermediários que fortalecem a argumentação da corretude do algoritmo. Tais resultados estão expostos
abaixo nesta seção e a formalização em PVS dos mesmos pode ser encontrado no repositório apresentado previamente na introdução deste texto.

\begin{lemma}
\label{merge-preserves-occurrences}
        Para quaisquer $\rho$ e $\eta$, listas, e $n$ valor, o número de ocorrências de $n$ em $MERGE(\rho,\ \eta)$ é igual ao número de ocorrências
        de $n$ em $\rho$ mais o número de ocorrências de $n$ em $\eta$.
\end{lemma}

\begin{proof}
        Fixando um $n$ qualquer, induzindo sobre o tamanho das listas de entrada e valendo-se da definição de $MERGE$, essencialmente,
        chegamos a dois casos em que a propriedade descrita no lema deve valer. O primeiro caso é quando algumas das listas de
        entrada são nulas, neste caso, o algoritmo retorna uma das listas de entrada que, por si só, mantém as ocorrências de
        qualquer valor $n$. O segundo caso ocorre quando ambas as listas possuem ao menos um elemento, neste caso, em algum momento será inserido
        na lista resultante todas as ocorrências de $n$ presentes nas listas de entradas.
\end{proof}

\begin{lemma}
\label{merge-preserves-length}
        Para quaisquer entradas $\rho$ e $\eta$, listas, o tamanho da lista resultante de $MERGE(\rho,\ \eta)$ é a soma
        dos tamanhos de $\rho$ e $\eta$.
\end{lemma}

\begin{proof}
        Induzindo sobre o tamanho da listas e utilizando da definição de merge, a afirmação do
        lema deve ser verificado em três situações:
        \begin{enumerate}
                \item $\rho$ ou $\eta$ são listas nulas.
                \item A cabeça de $\rho$ é menor ou igual que a cabeça de $\eta$.
                \item A cabeça de $\rho$ é maior que a cabeça de $\eta$.
        \end{enumerate}

        No primeiro caso, ao menos uma das listas tem tamanho zero, o algoritmo retorna então a concanetação de uma
        lista nula com outra lista $\psi$, que com certeza terá o mesmo tamanho que $\psi$. Para os outros dois casos
        basta demonstrar que tamanho de $\rho$ mais tamanho de $\eta$ é igual à $1 + MERGE(cdr(\rho),\ \eta)$ ou
        $1 + MERGE(\rho,\ cdr(\eta))$. 

        A indução sobre o tamanho das listas neste lema nos garante que se a soma dos tamanhos de duas listas $\psi$ e $\gamma$ 
        for estritamente menor que a soma dos tamanhos de $\rho$ e $\eta$, então o tamanho de $\psi$ mais tamanho de $\gamma$ é igual ao tamanho
        de $MERGE(\psi, \gamma)$. Podemos utilizar o seguinte resultado para os pares de lista de entrada $(cdr(\rho),\ \eta)$ 
        e para $(\rho, cdr(\eta))$ sem nenhum problema, uma vez que garantidamente nenhuma das listas é nula. Com isto, o problema
        é reduzido a verificação de
        \begin{equation*}
                LENGTH(\rho) + LENGTH(\eta) = 1 + LENGTH(cdr(\rho)) + LENGTH(\eta)
        \end{equation*}
        e,
        \begin{equation*}
                LENGTH(\rho) + LENGTH(\eta) = 1 + LENGTH(\rho) + LENGTH(cdr(\eta))
        \end{equation*}
        que são ambas, trivialmente, verdade.
\end{proof}

\begin{lemma}
\label{merge-is-permutation}
        Para quaisquer entradas $\rho$ e $\eta$, $MERGE(\rho,\ \eta)$ é uma permutação de concanetação de $\eta$ e $\rho$.
\end{lemma}
\begin{proof}
        Tomando $\rho$ e $\eta$ quaisquer, é sabído que a somas das ocorrências nas duas listas é igual às ocorrências
        na concanetação das mesmas. Logo, basta demonstrar que a soma das ocorrências em $\rho$ e $\eta$ é igual às
        ocorrências em $MERGE(\rho,\ \eta)$, o que efetivamente é o resultado verificado pelo lema \ref{merge-preserves-occurrences}. 
\end{proof}

\begin{lemma}
\label{merge-of-sorted-is-sorted}
        Para quaisquer entradas $\rho$ e $\eta$, se ambas as listas estão ordenadas então
        a resultante de $MERGE(\rho,\ \eta)$ também será uma lista ordenada. 
\end{lemma}

\begin{proof}
        Basta demonstrar que para qualquer iteração de $MERGE$, o menor elementos dentre ambas as listas é escolhido para inserção no final
        da lista parcialmente resultante. Uma vez que ambas as listas são ordenadas, necessariamente o menor elemento será a cabeça de $\rho$ ou de $\eta$.
        Utilizando da definição de $MERGE$ e induzindo sobre o tamanho de ambas as listas 3 situações emergem:
        \begin{enumerate}
                \item Alguma das listas $\rho$ e $\eta$ tem tamanho 0.
                \item A cabeça de $\rho$ é menor ou igual que a cabeça de $\eta$.
                \item A cabeça de $\rho$ é maior que a cabeça de $\eta$.
        \end{enumerate}

        Na primeira situação, no máximo uma das listas possui elementos, então o elemento tomado para a composição da lista resultante será a cabeça
        (o menor elemento) daquela que ainda não é vazia, por conveniência, o algoritmo já retorna a lista inteira, mas se de fato tivesse que escolher apenas
        um, seria o menor disponível. O segundo e o terceiro caso são similares, a cabeça de $\rho$ é igual ao menor elemento de $\rho$ e o mesmo vale para a cabeça
        de $\eta$, logo, o menor entre os ambos será o menor dentre todos e este será escolhido para a inserção ao final da lista resultante, a chamada recursiva 
        posterior cai novamente em algum dos 3 casos.

\end{proof}

\begin{lemma}
\label{mergesort-preserves-length}
        Para qualquer entrada $\tau$, lista, o tamanho de $\tau$ é igual ao tamanho de $MERGESORT(\tau)$.
\end{lemma}

\begin{proof}
        Induzindo sobre o tamanho da lista $\tau$ de entrada e valendo-se da definição de $MERGESORT$. Basta demonstrar
        que nos dois caminhos de execução do algoritmo não há modicação no tamanho da lista de entrada. O primeiro caso
        ocorre quando o tamanho da lista de entrada $\tau$ é menor ou igual à um, que é trivial, uma vez que a respota do algoritmo
        é a própria lista $\tau$.

        No caso da chamada recursiva, o lema \ref{merge-preserves-length} nos garante que nenhuma chamada à $MERGE$ 
        altera o tamanho da lista resultante e a própria parada de $MERGESORT$ conserva o tamanho da lista de entrada como visto
        anteriormente. Dessa forma, $MERGESORT$ conserva o tamanho da lista de entrada na lista resultante. 
\end{proof}

\begin{lemma}
\label{mergesort-is-permutation}
        Para qualquer lista $\tau$, a resultante de $MERGESORT(\tau)$ é uma permutação de $\tau$.
\end{lemma}

\begin{proof}
        A definição de permutação de uma lista nos diz que para qualquer chave $k$, o número de ocorrências
        de $k$ em $\tau$ é o mesmo que em $PERMUTATION(\tau)$. Utilizando da definição de $MERGESORT$ e aplicando
        indução sobre o tamanho da lista de entrada, obtemos dois casos em que faz necessário demonstrar que
        a saída do algoritmo é uma permutação da entrada. O primeiro caso, trivial, diz respeito à listas com tamanhos menores
        ou iguais à um, neste caso o algoritmo retorna a própria lista de entrada que por definição é uma permutação de si mesma.
        
        O segundo caso a ser demonstrado é para listas de tamanho maior do que um, entradas as quais o algoritmo realiza ao menos
        uma chamada recursiva a sí próprio. O lema \ref{merge-is-permutation} nos garante que a chamada à função $MERGE$
        efetivamente não altera o status de permutação de uma lista e o fato de o caso de parada da recursão de $MERGESORT$ também
        não modificar esta configuração nos permite concluir que $MERGESORT(\tau)$ é uma permutação de $\tau$ para qualquer $\tau$.
\end{proof}

\begin{lemma}
\label{mergesort-sorts}
        Para qualquer lista $\tau$, a resultante de $MERGESORT(\tau)$ é uma lista ordenada.
\end{lemma}

\begin{proof}
        Aplicando indução forte sobre o tamanho da lista $\tau$ chegamos a dois casos a serem demonstrados.
        Se a lista tem tamanho menor ou igual à um, a mesma está trivialmente ordenada. Caso contrário,
        o lema \ref{merge-of-sorted-is-sorted} nos garante e o merge das chamadas recursivas de $MERGESORT$
        para cada uma das metades de $\tau$, $prefix\ e\ suffix$,  tem como resultado uma lista ordenada
        desde que $MERGESORT(prefix)$ e $MERGESORT(suffix)$ sejam ambas listas ordenadas.

        Aqui podemos nos valer de propriedade desta indução forte que diz que qualquer lista $\phi$ tal que
        o tamanho da mesma seja estritamente menor que o tamanho de $\tau$, então $MERGESORT(\phi)$ é uma lista
        ordenada. Sabemos que $\tau$ tem tamanho maior ou igual que 2, logo qualquer metade desta lista
        terá tamanho estritamente menor que $\tau$.
\end{proof}

\begin{theorem}
\label{mergesort-is-correct}
        Para qualquer lista $\tau$, a resultante de $MERGESORT(\tau)$ é uma permutação ordenada de $\tau$.
\end{theorem}

\begin{proof}
        Aplicando os lemas \ref{mergesort-sorts} e \ref{mergesort-is-permutation}, a propriedade da corretude torna-se
        evidente.
\end{proof}

\section{Análise de complexidade do Merge Sort}
\label{complexity}
A complexidade de um algoritmo de ordenação se baseia em contar 
o número de comparações realizadas durante sua execução. 
Neste sentido, existe uma classificação de algoritmos de ordenação chamada
algoritmos de ordenação por comparação, que, segundo Cormen \cite{cormen2012algoritmos},
são algoritmos baseados em comparações entre dois elementos e 
efetuam pelo menos \textbf{\textit{$\Omega$(n lg n)}} comparações
no pior caso. 

O Merge Sort é um algoritmo de ordenação por comparação, e portanto,
sua análise de complexidade se baseia em contar quantas comparações
são efetuadas para ordenar uma lista de tamanho \textit{\textbf{n}}. Para
tanto, foram necessárias modificações no algoritmo, de forma a incluir
um contador de comparações. O pseudo-código abaixo representa a ideia 
da inclusão de um contador no Merge Sort:
$\newline$
\begin{algorithm}[H]
  \KwData{$\rho, \eta : Sorted\ lists,\ \sigma : natural $}
  \KwResult{$Sorted\ merge\ of\ \rho\ and\ \eta,\ number\ of\ comparisons$}
  \eIf{$\rho = Nil$ or $\eta = Nil$}{
      $\textbf{return}\  \rho + \eta,\ \sigma$ 
  }{
      \eIf{$car(\rho) \leq car(\eta)$}{
          $\textbf{return}\ car(\rho) + MERGE(cdr(\rho) ,\ \eta,\ \sigma + 1)$
      }{
          $\textbf{return}\ car(\eta) + MERGE(\rho,\ cdr(\eta),\ \sigma + 1)$
      }
  }
  \caption{COUNT-MERGE}
\end{algorithm}
$\newline$
$\newline$
\begin{algorithm}[H]
  \KwData{$\tau : List\ of\ comparable\ elements$}
  \KwResult{$Sorted\ permutation\ of\ \tau,\ number\ of\ comparisons $}
  \eIf{$length(\tau) \leq 1$}{
          $\textbf{return}\ \tau, \ 0 $
  }{
          $prefix,\ \sigma \leftarrow\ MERGESORT(first\_half(\tau))$\;
          $suffix,\ \iota \leftarrow\ MERGESORT(second\_half(\tau))$\;
          $\textbf{return}\ MERGE(prefix,\ suffix,\ \sigma + \iota)$\;  
  }
  \caption{COUNT-MERGESORT}
\end{algorithm}
$\newline$
$\newline$
A adição de um contador no Merge Sort permite a análise de comparações, porém,
não é possível assegurar que o contador não inseriu um erro no algoritmo. Portanto,
é necessário provar os seguintes lemas, que asseguram a equivalência entre o 
Merge Sort com e sem contador:

\begin{lemma}
    \label{cmerge_equiv_merge}
    $\forall$ $\rho$, $\eta$ listas de naturais, a resultante de $MERGE(\rho,\ \eta)$
    é igual à resultante de $COUNT-MERGE(\rho,\ \eta, \sigma)$, a não ser por $\sigma$.
\end{lemma}

\begin{lemma}
  \label{cmergesort_equiv_mergesort}
  $\forall$ $\rho$, lista de naturais, a resultante de $MERGESORT(\rho)$
  é igual à resultante de $COUNT-MERGESORT(\rho)$, a não ser pelo contador retornado.
\end{lemma}

\section{Conclusão}
\label{conclusion}

O trabalho desenvolvido foi desafiador no sentido de ter que se preocupar com pequenos detalhes de cada prova. 
Os lemas e as provas apresentados acima apesar de simples e por vezes evidentes levaram a formalizações
complexas e de dificil entendimento à olho nu, por vezes, uma simples enunciação correta mas não favorável
ao longo do trabalho tornou o desenvolvimento da prova impraticável. Isso traduziu-se em alguns lemas
sem uma prova completa ainda que a argumentação tenha oferecido uma boa base para a veracidade dos mesmos.

De fato tanto o PVS quanto outros assistentes de prova não são indicados para quem deseja aprender a fazer uma prova,
e sim para aqueles que já dominam a prova e não possuem algum porém em relação à fundamentação lógica por trás.
A conjunto de pessoas que utilizam assistente de provas ainda é muito restrito mas há um movimento para que isto cresça
atrelado com a evolução deste tipo de Software.

Em relação ao trabalho aqui apresentado, é deixada uma formalização completa da corretude do Merge Sort e um largo passo
em relação à formalização da complexidade do algoritmo em seu pior caso, ambas em PVS. As provas deixadas estão hospedadas
em modo público no repositório citado acima e podem ser verificadas e modificadas à vontade. Vale salientar que o uso
da Nasa PVS Library ou similar é imprencidível para a formalização da complexidade uma vez que abordagem aqui adotada utiliza
de propriedades não triviais da função logarítmica.

\bibliographystyle{acm}
\bibliography{reference}

\end{document}

%%% Local Variables:
%%% mode: latex
%%% TeX-master: t
%%% End:
